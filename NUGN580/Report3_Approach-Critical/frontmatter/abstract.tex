%
% File: abstract.tex
% Author: V?ctor Bre?a-Medina
% Description: Contains the text for thesis abstract
%
% UoB guidelines:
%
% Each copy must include an abstract or summary of the dissertation in not
% more than 300 words, on one side of A4, which should be single-spaced in a
% font size in the range 10 to 12. If the dissertation is in a language other
% than English, an abstract in that language and an abstract in English must
% be included.

\chapter*{Abstract}
\begin{SingleSpace}
\initial{C}riticality is usually derived from historical records of the fuel loading pattern. In some cases, one might not know when the criticality occurs. Typically, this is the case when refueling the reactor, either with a different loading pattern or with changes in some materials in the core area.

It can be dangerous to reach criticality before expected, thus one needs to know when to expect it. In order to do so, neutron counts from a detector can be used to determine the subcritical multiplication factor, which gives a good indication of the criticality state of the reactor.

After taking 15 fuel assemblies from the core, they were put back in a few at a time. The subcritical multiplication factor was computed, allowing for a good prediction of the number of assemblies needed in the core to go critical. At the GSTR facility, a full core minus one fuel rod was enough to get to criticality.
\end{SingleSpace}
\clearpage
