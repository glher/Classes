%
% File: conclusion.tex
%
\let\textcircled=\pgftextcircled
\chapter{Conclusion}
\label{chap:conclusion}

\initial{T}he USGS TRIGA research nuclear reactor (GSTR) have been used to irradiate a sample in order to determine its isotopic composition. While this method can also give the quantities (mass) of each nucleide in the sample, only an energy calibration of the spectrometer was performed, rendering this information unavailable. Instead, the elements in the sample have been identified, using the activated sample activity and a spectrometry.

The sample was found to have a 36 minutes half-life. The spectrometry generated low confidence data, due to potential ongoing work on the library. Nonetheless, if the results from the measurements are considered correct within the usual 1 keV uncertainty, it has been derived that the sample contained Hafnium and Gadolinium mostly, with an accumulation of Tantalum probably caused by the activation of Hafnium.

%=======

