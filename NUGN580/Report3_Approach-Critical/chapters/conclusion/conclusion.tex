%
% File: conclusion.tex
%
\let\textcircled=\pgftextcircled
\chapter{Conclusion}
\label{chap:conclusion}

\initial{A}n approach to criticality was performed on the USGS TRIGA reactor core, from a fuel mass point of view. Fifteen fuel elements were taken out of the core and progressively put back in while surveilling the neutron count rate to predict the number of fuel elements needed to reach criticality.

It was seen that the approach taken was conservative, the prediction of the number of fuel rods to reach criticality being systematically underestimated. This has to do with the reloading pattern. Indeed, if by chance the least reactive fuel rods had all been inserted first, the operators could have assumed that they were far from criticality and decided to speed up the process by reloading more of the most reactive fuel elements at once, thus achieving criticality earlier than anticipated.

A full core was not necessary to obtain a critical mass, since the core went critical with one fuel element, worth 75 cents, to spare.

%=======

