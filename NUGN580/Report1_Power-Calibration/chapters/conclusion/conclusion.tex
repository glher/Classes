%
% File: conclusion.tex
%
\let\textcircled=\pgftextcircled
\chapter{Conclusion}
\label{chap:conclusion}

\initial{A}ll three detectors in the GSTR (NM1000, NPP1000 and NP1000) display a lower than measured thermal power output (respectively 840, 870 and 850 kW versus 910 kW). This discrepancy can of course be caused by badly calibrated instruments. In that case, the three detectors will need to be re-calibrated soon in order to give the operators the correct thermal power output, and not be in violation of the GSTR license. Other causes could be thought of to explain this difference.

One of the potential other causes was the presence of a quite new 8 inches beam tube, which might have been isolating some water and thus lower the actual amount of water in the tank. However, a quick calculation shows that this is unlikely to be the cause, at least by itself. Another potential cause for the consequent discrepancy could be a lack of luck with the temperature probes, the uncertainties at the extreme end of the range being able to account for some of the difference. That would also be quite unlikely to be the sole reason, since most values for both detectors are in agreement with one another.

The second calibration, which was performed the next day at the facility brings some more light to this power calibration. It was done in different conditions (xenon, ambient humidity and heat, etc), yet it showed similar results, all power instruments reading lower than the actual measured power. The alternative explanations can thus be invalidated. The power channels were adjusted and are now all correctly calibrated.

In regards of the present analysis, it remains to be seen how to be more confident in the data collected, so as to not have needed a re-run of the calibration. In our case, a longer data point gathering could have helped, as well as adding another probe to lower the measurement uncertainties. The reactor tank water was still a way off of the 60\degree C limit, and more points could have been measured had we not been contrained by time.


%=======

