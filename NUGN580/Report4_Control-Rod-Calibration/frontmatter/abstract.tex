%
% File: abstract.tex
% Author: V?ctor Bre?a-Medina
% Description: Contains the text for thesis abstract
%
% UoB guidelines:
%
% Each copy must include an abstract or summary of the dissertation in not
% more than 300 words, on one side of A4, which should be single-spaced in a
% font size in the range 10 to 12. If the dissertation is in a language other
% than English, an abstract in that language and an abstract in English must
% be included.

\chapter*{Abstract}
\begin{SingleSpace}
\initial{T}he differential and integral worth of the control rods must be measured and calibrated from time to time, to check for the rods efficiency. Several limits must be respected, such as the maximum core reactivity, the maximum pulse reactivity, the maximum reactivity insertion rate and, for research reactors, the maximum experiment reactivity. This is done by using the reactivity values of the control rods, hence the utmost importance of having precise values for these parameters. In this project, the positive period method is used, consisting of measuring the reactivity brought by moving up the calibrated control rod in discrete steps, until it is completely out of the core.

In power reactors, various methods are used. For example, an Exchange-Dilution method can sometimes be used for reactor with boron water.

The procedure for this experiment will be to get to a critical, stable state with the control rod to calibrate in its down position. Then, it will be taken up a discrete amount of steps, and the period will be to get a twenty-fold increase in power (from 20W to 400W) will be measured. Using the delayed neutron fractions and the period in the Inhour equation, the reactivity will be calculated.

The results show that the transient control rod is relatively well calibrated, vertically centered in the core. It also demonstrates that the licensing limits for this reactor are comfortably respected.
\end{SingleSpace}
\clearpage
