%
% File: conclusion.tex
%
\let\textcircled=\pgftextcircled
\chapter{Conclusion}
\label{chap:conclusion}

\initial{A} rod calibration was performed on the USGS TRIGA reactor core, for the pulse (or transient) rod. Only the pulse rod was moved up, while the other three rods were used to stabilize the reactor in a critical state after each steps. The neutron delayed fraction and characteristics, along with the reactor period, were used to compute the differential and thus integral reactivity worth of the calibrated rod. This allowed the operator to see that the transient rod could be considered well vertically-centered in the core, if potentialy slightly higher than nominal. It can be noted that the experiment was facilitated by the absence of Xenon poison in the core.

Moreover, it showed that the licensing limits were still respected, notably the maximum pulse reactivity and the maximum reactivity insertion rate.


%=======

