%
% File: conclusion.tex
%
\let\textcircled=\pgftextcircled
\chapter{Conclusion}
\label{chap:conclusion}

\initial{E}ven though the usefulness of pulsing a nuclear reactor is debatable, and confined to very specific irradiation tasks, the fact that it is possible while respecting the safety of the reactor is quite remarkable. This project showed that the estimates used for the stotal heat of the core were relatively good when applied to low prompt reactivity insertion, but degraded the higher it got. The opposite effect was observed for the prompt negative temperature coefficient.

Overall however, the Fuchs-Nordheim model is shown to be in good agreement with the measured data, with the correct correlations between the prompt reactivity insertion and the different parameters of the core (peak power, period, shape of the power time distribution, ...).

The maximum pulse value for the GSTR is \$2.3, due to the worth of the transient rod and the reactivity of its neighboring fuel elements. The GSTR is licensed for a pulse at \$3. The reactor was still able to pulse at 1.3 GW for a short amount of time.


%=======

