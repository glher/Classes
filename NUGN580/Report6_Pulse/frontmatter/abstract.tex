%
% File: abstract.tex
% Author: V?ctor Bre?a-Medina
% Description: Contains the text for thesis abstract
%
% UoB guidelines:
%
% Each copy must include an abstract or summary of the dissertation in not
% more than 300 words, on one side of A4, which should be single-spaced in a
% font size in the range 10 to 12. If the dissertation is in a language other
% than English, an abstract in that language and an abstract in English must
% be included.

\chapter*{Abstract}
\begin{SingleSpace}
\initial{W}hen the positive reactivity added to a critical core is equal to the effective delayed neutron franction, $\beta_{eff}$, prompt criticality is achieved. That corresponds to a positive reactivity of $1\$$. Super prompt critical states arise when the reactivity insertion is above this one dollar limit. Essentially, in this state of operation, the delayed neutrons do not control the reactor period, and the power grows exponentially.

TRIGA reactors such as the GSTR are able to withstand this power excursion by design, due to the decrease in the thermal non-leakage probability caused by the fuel temperature effect. From pulsing, it is possible to compute the prompt neutron lifetime, the negative temperature coefficient and the specific heat of the core. The Fuchs-Nordheim model approximation will be tested against the experimental values obtained.

\end{SingleSpace}
\clearpage
