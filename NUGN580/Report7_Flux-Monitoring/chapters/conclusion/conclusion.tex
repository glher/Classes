%
% File: conclusion.tex
%
\let\textcircled=\pgftextcircled
\chapter{Conclusion}
\label{chap:conclusion}

\initial{W}e utilized Gamma Spectrometry to irradiate high purity NaCl in order to ascertain flux values across the profile of the reactor in the Central Thimble. We irradiated 8 samples of salt for five minutes, then a day later measured the activity of the $Na_{24}$ to determine flux values. By waiting a day, the radioactive Cl, due to its short half life, decayed away into a stable state, thus allowing us to focus solely on the sodium. The low mass of salt we loaded into the core allowed us to use such a short irradiation time.

Our calculations determined that flux was maximized 58 mm below the center of the Central Thimble. A maximum flux value of $4.40 * 10^{12}$ was calculated. This varies by a decade from our expected maximum of approximately $10^{13}$. A maximum activity of 0.296 $\micro$Ci was measured using the Gamma Spectrometer. 

Due to some amount of measurement errors, we had to disregard two of our data points, which tremendously skewed our final results. We expect that accurate measurements at these two positions would have given us more insight into the flux profile across the central thimble. As these two positions were near to the center of the thimble, we would expect them to have high values of flux, and as such, we would expect a more symmetric flux profile across the central line. Thus, we must recommend that this lab be repeated in order to obtain more valid measurements.


%=======

