%
% File: conclusion.tex
%
\let\textcircled=\pgftextcircled
\chapter{Conclusion}
\label{chap:conclusion}

\initial{T}he project targeted by this report was to qualify the role of Xenon, a fission product neutron absorber, in the reactivity present in the core. In order to do so, the core was run at full power for a period of eight hours, allowing for Xenon buildup, though far from equilibrium concentration, and then shut down. During the shut down state, a Xenon buildup happens from the decay of Iodine. The goal is to detect the amount of antireactivity caused by the Xenon, and its evolution, when the reactor is started up again the next day.

This shows that the Xenon population indeed increases during the shutdown. Due to logistical reasons (time), the Xenon peak cannot be observed, and has to be inferred from theory. This gives an estimated maximum Xenon antireactivity of around 1.25\$ 15 hours after the initial startup, 7 hours after the shutdown. The antireactivity then decreases with the flux, considering that the GSTR has more than enough excess reactivity to overcome the presence of Xenon at the time of start-up the next day.

Had we tried to start up the reactor 7 hours after running it at full power for two days straight, it is likely that the core could not have gone critical.


%=======

