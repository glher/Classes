%
% File: chap01.tex
%
\let\textcircled=\pgftextcircled
\chapter{Reactor Heat Generation}
\label{chap:intro}

\initial{S}everal exercises from the book written by M. M. El Wakil~\cite{book01} are tackled in this homework. The problems in this section relate to the fourth chapter of the book, covering the subject of the reactor heat generation.

%=======
\section{Uranyl Sulphate}
\label{prob41}

\subsection{Problem}
\textit{Calculate the number density of $U^{235}$, nuclei/$cm^{3}$, in a 10 percent enriched uranyl sulphate ($UO_2SO_4$) in a 50 percent by mass aqueous solution at $500F$. Assume density of solution to be equal to that of water.}

\subsection{Solution}

We know that in order to obtain the fissionable fuel density $N_{ff}$, we can use Equation~\ref{eq41}.

\begin{equation}\label{eq41}
N_{ff} = \frac{N_A}{M_{ff}}\rho_{ff}
\end{equation}

Where,
\begin{conditions}
M_{ff} & Molecular mass of fissionable fuel used \\
N_A & Avogadro number \\
\rho_{ff} & Density of fissionable fuel used
\end{conditions}

In this equation, we know $M_{ff}$ and $N_A$. We thus have to compute $\rho_{ff}$, using Equation~\ref{eq42}.


\begin{equation}\label{eq42}
\rho_{ff} = rf\rho_{fm}
\end{equation}

Where,
\begin{conditions}
f & Mass fraction of the fuel in the fuel material \\
r & Enrichment \\
\rho_{fm} & Density of fuel material
\end{conditions}

The mass fraction is given by Equation~\ref{eq43}.


\begin{equation}\label{eq43}
f = \frac{rM_{ff} + (1-r)M_{nf}}{rM_{ff} + (1-r)M_{nf} + 6*M_{O} + M_S} = 0.650
\end{equation}

The density of the solution being taken equal to that of water, $1\ g.cm^{-3}$, and the solution being composed at 50\% mass of the fuel material, knowing the density of water in our case would allow us to compute the density of the fuel material. Using some water tables, we can obtain the density of saturated water at 500F, $\rho = 0.784\ g.cm^{-3}$. Thus, this would give us a density of the fuel material equal to $\rho_{fm} = 1.216\ g.cm^{-3}$.

Consequently, we can calculate $N_{ff} = \text{\num{2.02e20}}\ cm^{-3}$.

\section{Fission cross section}
\label{prob42}

\subsection{Problem}
\textit{Calculate the effective fission cross section of 3 percent enriched $UO_2$ if it is used in an ordinary water-moderated thermal reactor. The core lattice arrangement is such that the fuel/moderator ratio is 1:1.9 by volume. Assume the moderator is at 500F and 2000 psia, and that the other reactor-core materials contribute 2 barns of 1/V absorption per atom of hydrogen.}

\subsection{Solution}

The effective fission cross section is given by Equation~\ref{eq44}.


\begin{equation}\label{eq44}
\bar{\sigma_{f}} = 0.8862 f(T) \sigma_{f,0} \sqrt{\frac{T_0}{T}}
\end{equation}

For $T = 500 F = 533 K$, $f(T) = 0.93$ and $\sigma_{f, 0} = 577.1\ b$, we can calculate $\bar{\sigma_f} = 352.7\ b$.


\section{Volumetric thermal source}
\label{prob43}

\subsection{Problem}
\textit{Calculate the volumetric thermal source strength, $Btu.h^{-1}.ft^{-3}$ at a point 49.9\% of the radial distance and halfway above the centerplane of a cylindrical, homogeneous bare reactor core containing enriched $UO_2SO_4$ in solution in $H_2O$. The density of the fuel is 0.255 $g.cm^{-3}$. The temperature of the solution is 533K. The enrichment is 10 percent. The maximum neutron flux in the core is \num{1e14}.}

\subsection{Solution}

$N_{ff}$ can be calculated for the fuel at hand using Equations~\ref{eq41} and~\ref{eq42}. Using the density of the fuel given ($\rho_f = 0.255\ g.cm^{-3}$), we can obtain Equation~\ref{eq45}.

\begin{equation}\label{eq45}
N_{ff} = \frac{N_A}{235.0439}e\rho_{f} = \text{\num{6.533e19}}\ cm^{-3}
\end{equation}

The fission cross section for the fuel is given by problem~\ref{prob42}. We obtain $\bar{\sigma_{f}} = 352.7\ b$. If we assume the energy per fission to be $G = 180\ MeV$, we can use Equation~\ref{eq46}.

\begin{equation}\label{eq46}
q''' = G N_{ff} \bar{\sigma_{f}} \phi(r_1, z_1)
\end{equation}

The only unknown left in Equation~\ref{eq46} is the flux at the given point. It can be calculated using Equation~\ref{eq47}, with $z_1 = H/4$ and $r_1 = 49.9*R/100$.

\begin{equation}\label{eq47}
\phi(r_1, z_1) = \phi_0 \cos(\frac{\pi}{4}) J_0(\frac{49.9}{100}2.405) = \phi_0 * \frac{1}{sqrt{2}} * 0.671 = \text{\num{4.75e13}}
\end{equation}

Using a conversion constant ($1\ MeV.s-1.cm^{-3} = \text{\num{1.5477e-8}}\ Btu.h^{-1}.ft^{-3}$), this gives us $q''' = \text{\num{3.049e6}}\ Btu.h^{-1}.ft^{-3}$.



\section{Total heat generation}
\label{prob44}

\subsection{Problem}
\textit{A fast homogeneous fluid-fueled reactor uses uranium in solution in molten Bismuth so that the fuel concentration is \num{1e20} $U^{235}\ cm^{-3}$. The reactor core is 5 feet in diameter and 8 feet high. The extrapolated height is 12 feet. Because of strong radial reflection, the neutron flux may be considered uniform in the radial direction. The fast neutron flux at the core entrance (bottom) is \num{1e15}. Take $\bar{\sigma_f}$ for fast neutrons as 5 barns. Calculate the total heat generated in the core.}

\subsection{Solution}

In a cylindrical core, we can write Equation~\ref{eq48}.

\begin{equation}\label{eq48}
q''' = G N_{ff} \bar{\sigma_{f}} \phi_0 \cos(\frac{\pi z}{H_e}) J_0(\frac{2.405r}{R_e})
\end{equation}

The definition of a radially uniform flux eluding me, I will assume that it implies a constant radial flux $S = J_0(0) = 1$, so that Equation~\ref{eq48} becomes Equation~\ref{eq49}.

\begin{equation}\label{eq49}
q''' = G N_{ff} \bar{\sigma_{f}} \phi_0 \cos(\frac{\pi z}{H_e})
\end{equation}

The total heat generated in the core is given by $Q = \int \Delta Q$, where $\Delta Q = q''' \Delta V$, $V$ the volume, as seen in Equation~\ref{eq410} and~\ref{eq411}.


\begin{equation}\label{eq410}
\Delta Q = G N_{ff} \bar{\sigma_{f}} \phi_0 \cos(\frac{\pi z}{H_e}) 2\pi r\Delta r \Delta z
\end{equation}


\begin{equation}\label{eq411}
Q = G N_{ff} \bar{\sigma_{f}} \phi_0 2\pi \int_0^{R} r dr \int_{-H/2}{H/2} \cos(\frac{\pi z}{H_e}) dz
\end{equation}

Solving the integrals, we obtain Equation~\ref{eq412}

\begin{equation}\label{eq412}
Q = G N_{ff} \bar{\sigma_{f}} \phi_0 2\pi \frac{R^2}{2}  \frac{H_e\sqrt{3}}{\pi} = G N_{ff} \bar{\sigma_{f}} \phi_0 R^2 H_e\sqrt{3}
\end{equation}

Considering the following values:

\begin{conditions}
G & 180 MeV \\
N_{ff} & \num{1e20} $cm^{-3}$\\
\phi_0 & \num{1e15} $cm^{-2}.s^{-1}$ \\
\bar{\sigma_{f}} & \num{5e-24} $cm^{2}$
\end{conditions}

We can obtain the total heat generated $Q = \text{\num{2.21e20}}\ MeV.s^{-1} = \text{\num{1.21e8}}\ Btu.hr^{-1}$

\section{Minimum critical volume}
\label{prob45}

\subsection{Problem}
\textit{Find the minimum critical volume of a cylindrical bare reactor.}

\subsection{Solution}

The buckling for a cylindrical core, with extrapolated radius and height $R_e$ and $H_e$, was computed during the third homework. We recall Equation~\ref{eq413}.


\begin{equation}\label{eq413}
B^2 = \left( \frac{2.405}{R_e} \right)^2 + \left( \frac{\pi}{H_e} \right)^2
\end{equation}

In this case, we will assume that $R = R_e$ and $H = H_e$. Then, we get Equation~\ref{eq414}.

\begin{equation}\label{eq414}
B^2 = \left( \frac{2.405}{R} \right)^2 + \left( \frac{\pi}{H} \right)^2
\end{equation}

We can subsitute this into the volume, as seen in Equation~\ref{eq415}.


\begin{equation}\label{eq415}
V = \pi R^2 H = \pi 2.405^2 * \frac{H}{B^2H^2-\pi^2}
\end{equation}

The minimum critical volume is given when the derivative of $V$ with respect to $H$ is zero, Equation~\ref{eq416}.


\begin{equation}\label{eq416}
\frac{dV}{dH} = \frac{2.405^2 H^2(\pi B^2H^2 - 3\pi^3)}{(\pi^2 - B^2H^2)^2} = 0
\end{equation}

This is possible if and only if $H^2 = \frac{3\pi^3}{\pi B^2} = 3\frac{\pi^2}{B^2}$, and hence $H = \frac{\sqrt{3}\pi}{B}$. In that case, we can compute $R^2 = \frac{2.405^2}{B^2-\frac{\pi^2 B^2}{3\pi^2}}$, and hence $R = \sqrt{\frac{3}{2}}\frac{2.405}{B}$.

Plugging this back into the volume equation (Equation~\ref{eq415}), we obtain $V_{min} = \frac{148.3}{B^3}$.
