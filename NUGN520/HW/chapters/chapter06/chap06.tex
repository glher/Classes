%
% File: chap01.tex
%
\let\textcircled=\pgftextcircled
\chapter{Reactor Heat Generation}
\label{chap:intro}

\initial{S}everal exercises from the book written by M. M. El Wakil~\cite{book01} are tackled in this homework. The problems in this section relate to the fifth chapter of the book, covering the subject of heat conduction in reactor elements.

%=======
\section{[G1] - Fine mesh cruciform}
\label{prob51}

\subsection{Problem}
\textit{Calculate the temperature in a cruciform with a mesh $\Delta x = \Delta y = 0.1$.}

\subsection{Solution}

We can define the origin, taken at the center of the cruciform, to be $T(0,0)$. By symmetry, we know that $T(i,j) = T(-i,j)$ and $T(i,j) = T(i,-j)$. Consequently, the solution will be complete when $T(i,j), i \in [0, 11], j \in [0,1]$ are computed. By symmetry, we also know that $T(0,1) = T(1,0)$, so we can get rid of $T(0,1)$ in our computation.

We can approximate the temperature at these discrete points by using Equation~\ref{eq61}, knowing that $\Delta x = \Delta y$.

\begin{equation}\label{eq61}
T_n = \frac{T_{n+\Delta x} + T_{n-\Delta x} + T_{n+\Delta y} + T_{n-\Delta y}}{4} + \frac{\Delta T_g}{2}
\end{equation}

Where, in the case of uniform heat generation:

\begin{conditions}
\Delta T_g & $\frac{(\Delta x)^2 q'''}{2k}$
\end{conditions}

In our case, we consider $q''' = \text{\num{1e7}}$ and $k = 19.84$. Consequently, $\Delta T_g = 17.5\ {}^\circ F$. We also consider a surface temperature of $600\ {}^\circ F$.

Our goal is to solve the matrix Equation~\ref{eq62} for $x$.

\begin{equation}\label{eq62}
A . x = b
\end{equation}

We can calculate a few values in order to construct the matrice $A$ and $b$, using Equation~\ref{eq61}. We can obtain:

\begin{equation}\label{eq63}
T(0,0) = T(1,0) + \frac{\Delta T_g}{2} \implies T(0,0) - T(1,0) = \frac{\Delta T_g}{2}
\end{equation}

\begin{equation}\label{eq64}
T(1,0) = \frac{T(0,0) + T(2,0) + 2T(1,1)}{4} + \frac{\Delta T_g}{2} \implies T(1,0) - \frac{T(0,0)}{4} - \frac{T(2,0)}{4} - \frac{T(1,1)}{2} = \frac{\Delta T_g}{2}
\end{equation}

\begin{equation}\label{eq65}
T(1,1) = \frac{2T(1,0) + 2T(2,1)}{4} + \frac{\Delta T_g}{2} \implies T(1,1) - \frac{T(1,0)}{2} - \frac{T(2,1)}{2} = \frac{\Delta T_g}{2}
\end{equation}

\begin{equation}\label{eq66}
T(2,0) = \frac{T(1,0) + T(3,0) + 2T(2, 1)}{4} + \frac{\Delta T_g}{2} \implies T(2,0) - \frac{T(1,0)}{4} - \frac{T(3,0)}{4} - \frac{T(2,1)}{2} = \frac{\Delta T_g}{2}
\end{equation}

\begin{equation}\label{eq67}
\begin{aligned}
T(2,1) = \frac{T(1,1) + T(3,1) + T_S + T(2,0)}{4} + \frac{\Delta T_g}{2} \\ \implies T(2,1) - \frac{T(1,1)}{4} - \frac{T(3,1)}{4} - \frac{T(2,0)}{4} = \frac{T_S}{4} + \frac{\Delta T_g}{2}
\end{aligned}
\end{equation}

$T(i,0)\ \text{and}\ T(i,1)\ \text{for}\ i \in [2,10]$ will be identical, given the geometry of the problem.

\begin{equation}\label{eq68}
T(11,0) = \frac{T(10,0) + 2T(11,1) + T_S}{4} + \frac{\Delta T_g}{2} \implies T(11,0) - \frac{T(11,1)}{2} - \frac{T(10,0)}{4} = \frac{T_S}{4} + \frac{\Delta T_g}{2}
\end{equation}

\begin{equation}\label{eq69}
T(11,1) = \frac{T(11,0) + T(10,1) + 2T_S}{4} + \frac{\Delta T_g}{2} \implies T(11,1) - \frac{T(11,0)}{4} - \frac{T(10,1)}{4} = \frac{T_S + \Delta T_g}{2}
\end{equation}

Consequently, we can write $A$:

\setcounter{MaxMatrixCols}{30}
\tiny
\[
A_{0 \rightarrow 11}=
  \begin{bmatrix}
1     & -1    & 0     & 0     & 0     & 0     & 0     & 0     & 0     & 0     & 0     & 0     & 0     & 0     & 0 &     \\
-0.25 & 1     & -0.5  & -0.25 & 0     & 0     & 0     & 0     & 0     & 0     & 0     & 0     & 0     & 0     & 0 &     \\
0     & -0.5  & 1     & 0     & -0.5  & 0     & 0     & 0     & 0     & 0     & 0     & 0     & 0     & 0     & 0 &      \\
0     & -0.25 & 0     & 1     & -0.5  & -0.25 & 0     & 0     & 0     & 0     & 0     & 0     & 0     & 0     & 0 &      \\
0     & 0     & -0.25 & -0.25 & 1     & 0     & -0.25 & 0     & 0     & 0     & 0     & 0     & 0     & 0     & 0 &      \\
0     & 0     & 0     & -0.25 & 0     & 1     & -0.5  & -0.25 & 0     & 0     & 0     & 0     & 0     & 0     & 0 & \dots     \\
0     & 0     & 0     & 0     & -0.25 & -0.25 & 1     & 0     & -0.25 & 0     & 0     & 0     & 0     & 0     & 0 &      \\
0     & 0     & 0     & 0     & 0     & -0.25 & 0     & 1     & -0.5  & -0.25 & 0     & 0     & 0     & 0     & 0 &      \\
0     & 0     & 0     & 0     & 0     & 0     & -0.25 & -0.25 & 1     & 0     & -0.25 & 0     & 0     & 0     & 0 &      \\
0     & 0     & 0     & 0     & 0     & 0     & 0     & -0.25 & 0     & 1     & -0.5  & -0.25 & 0     & 0     & 0 &      \\
0     & 0     & 0     & 0     & 0     & 0     & 0     & 0     & -0.25 & -0.25 & 1     & 0     & -0.25 & 0     & 0 &      \\
0     & 0     & 0     & 0     & 0     & 0     & 0     & 0     & 0     & -0.25 & 0     & 1     & -0.5  & -0.25 & 0 &      \\
  \end{bmatrix}
\]




\tiny
\[
A_{12 \rightarrow 22}=
  \begin{bmatrix}
     & 0     & -0.25 & -0.25 & 1     & 0     & -0.25 & 0     & 0     & 0     & 0     & 0     & 0     & 0     & 0     \\
     & 0     & 0     & -0.25 & 0     & 1     & -0.5  & -0.25 & 0     & 0     & 0     & 0     & 0     & 0     & 0     \\
     & 0     & 0     & 0     & -0.25 & -0.25 & 1     & 0     & -0.25 & 0     & 0     & 0     & 0     & 0     & 0     \\
     & 0     & 0     & 0     & 0     & -0.25 & 0     & 1     & -0.5  & -0.25 & 0     & 0     & 0     & 0     & 0     \\
     & 0     & 0     & 0     & 0     & 0     & -0.25 & -0.25 & 1     & 0     & -0.25 & 0     & 0     & 0     & 0     \\
 \dots    & 0     & 0     & 0     & 0     & 0     & 0     & -0.25 & 0     & 1     & -0.5  & -0.25 & 0     & 0     & 0     \\
     & 0     & 0     & 0     & 0     & 0     & 0     & 0     & -0.25 & -0.25 & 1     & 0     & -0.25 & 0     & 0     \\
     & 0     & 0     & 0     & 0     & 0     & 0     & 0     & 0     & -0.25 & 0     & 1     & -0.5  & -0.25 & 0     \\
     & 0     & 0     & 0     & 0     & 0     & 0     & 0     & 0     & 0     & -0.25 & -0.25 & 1     & 0     & -0.25 \\
     & 0     & 0     & 0     & 0     & 0     & 0     & 0     & 0     & 0     & 0     & -0.25 & 0     & 1     & -0.5  \\
     & 0     & 0     & 0     & 0     & 0     & 0     & 0     & 0     & 0     & 0     & 0     & -0.25 & -0.25 & 1        
  \end{bmatrix}
\]

\normalsize
And we can also write $b$:

\tiny
\[
b =
  \begin{bmatrix}
8.75 \\
8.75 \\
8.75 \\
8.75\\
158.75\\
8.75\\
158.75\\ 
8.75\\
158.75\\ 
8.75\\
158.75\\ 
8.75 \\
158.75 \\
8.75 \\
158.75\\
8.75 \\
158.75 \\
8.75 \\
158.75 \\
8.75\\
158.75\\ 
158.75 \\
308.75
   \end{bmatrix}
\]

\normalsize


We can now solve Equation~\ref{eq62}, using Python. The python script is given in~\ref{py1}. We obtain the following results (Equation~\ref{eq610}):

\begin{equation}\label{eq610}
\begin{aligned}
T(0,0) = 723.8 \\
T(1,0) = 715.1 \\
T(1,1) = 703.2 \\
T(2,0) = 695.1 \\
T(2,1) = 673.8 \\
T(3,0) = 682.5 \\
T(3,1) = 662.0 \\
T(4,0) = 675.9 \\
T(4,1) = 656.8 \\
T(5,0) = 672.5 \\
T(5,1) = 654.3 \\
T(6,0) = 670.6 \\
T(6,1) = 652.9 \\
T(7,0) = 668.9 \\
T(7,1) = 651.8 \\
T(8,0) = 666.7 \\
T(8,1) = 650.2 \\
T(9,0) = 662.5 \\
T(9,1) = 647.2 \\
T(10,0) = 654.0 \\
T(10,1) = 641.1 \\
T(11,0) = 636.3 \\
T(11,1) = 628.1
\end{aligned}
\end{equation}


\newpage
\subsubsection{Python script}
\label{py1}
\begin{lstlisting}[language=python]
import numpy as np

m = 23
dx = 0.1
deltax, k, qtriple = (dx/12), 19.84, 1.e7
ts = 600.
dtg = (0.5*deltax**2)*qtriple/k

amat = np.zeros((m,m))
b = np.zeros(m)
    
amat[0,0] = 1.
amat[0,1] = -1.
b[0] = dtg/2.

amat[1,0] = -0.25
amat[1,1] = 1.
amat[1,2] = -0.5
amat[1,3] = -0.25
b[1] = dtg/2.


amat[2,1] = -0.5
amat[2,2] = 1.
amat[2,4] = -0.5
b[2] = dtg/2.

for i in range(3, m, 2):
    try:
        amat[i,i-2] = -0.25
        amat[i,i] = 1.
        amat[i,i+1] = -0.5
        amat[i,i+2] = -0.25
        b[i] = dtg/2.
    except IndexError:
        break

for i in range(4, m, 2):
    try:
        amat[i,i-2] = -0.25
        amat[i,i-1] = -0.25
        amat[i,i] = 1.
        amat[i,i+2] = -0.25
        b[i] = ts/4. + dtg/2.
    except IndexError:
        break

amat[m-1,m-1] = 1.
amat[m-1,m-2] = -0.25
amat[m-1,m-3] = -0.25
b[m-1] = ts/2. + dtg/2.


amat[m-2,m-2] = 1.
amat[m-2,m-1] = -0.5
amat[m-2,m-4] = -0.25
b[m-2] = ts/4. + dtg/2.

ysol = np.linalg.solve(amat,b)

print('Temperature solution:')

temp = []
for i in range(12):
    for j in range(2):
        if i == 0 and j == 1:
            continue
        temp.append("T(%s,%s)" % (i, j))
        
for i,t in enumerate(temp):
    print("%s = %.1f" % (t, ysol[i]))
    
# check of solution: 
if not np.allclose(np.dot(amat, ysol), b):
    print("Solution does not match!")
\end{lstlisting}
\newpage
