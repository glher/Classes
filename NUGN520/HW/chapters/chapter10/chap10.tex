%
% File: chap01.tex
%
\let\textcircled=\pgftextcircled
\chapter{Heat transfer and fluid flow, nonmetallic coolants}
\label{chap:intro}

\initial{S}everal exercises from the book written by M. M. El Wakil~\cite{book01} are tackled in this homework. The problems in this section relate to the tenth chapter of the book, covering the subject of heat transfer and fluid flow, for metallic coolants.

%=======
\section{[10-1] - Turbulent coolant tube}
\label{prob101}


\subsection{Problem}
\textit{A hypothetical liquid metal has $Pr = 0$ and $k = 50\ Btu.h^{-1}.ft^{-1}.{}^\circ F^{-1}$. It flows through a 1-in.-diameter tube with a Reynolds number of $1000000$. The tube-wall temperature is $1000{}^\circ F$. The temperature of the fluid halfway between the wall and the centerline is $900{}^\circ F$. Find (a) the temperature at the centerline of the tube, (b) the bulk temperature of the fluid and (c) the theat flux in $Btu.h^{-1}.ft^{-2}$.}

\subsection{Solution}

Martinelli provides solutions for the temperature gradient in pipe flow for various Reynolds and Prandtl number (and conveniently, $Re = 10^6$ and $Pr = 0$ is one of these solutions), seen in Figure 9-3 of the book~\cite{book01}.

We thus obtain, for $Pr=0$ and $Re = 10^6$:

\begin{equation}
\frac{T_w - T(y)}{T_w-T_c} = \frac{y}{r_0}
\end{equation}

Knowing $T_w = 1000{}^\circ F$ and also knowing $T(y=r_0/2) = 900{}^\circ F$, we can now easily solve the next equation for $T_c$:

\begin{equation}
\frac{T_w - T(y=r_0/2)}{T_w-T_c} = \frac{1}{2}
\end{equation}

\begin{equation}
T_c = 2*(900+500-1000) = 800{}^\circ F
\end{equation}

This is expected, we have a linear profile. In this case, the bulk temperature can be taken to be the average temperature in the channel, thus $T_f = 900{}^\circ F$. The book also states that (p. 236) "in highly turbulent flow, the temperature profile is fairly flat  over much of the cross-section, and the bulk temperature is taken as equal to the temperature at the center of the channel", so $T_f$ could also be taken as $800{}^\circ F$. Alternatively, the bulk temperature can be obtained from Figure 10-12. However, this requires to know the value of $\frac{q'''r_0^2}{k}$. This value can be obtained from Figure 10-11, by taking the temperature at the wall. Indeed:

\begin{equation}
10*\frac{T_w - T_c}{q'''r_0^2/k} = 2 \implies \frac{q'''r_0^2}{k} = \frac{T_w - T_c}{0.2} = 1000
\end{equation}

Plugging this into the data for Figure 10-12, we get:

\begin{equation}
\frac{T_w - T_f}{q'''r_0^2/k} = 10^{-2} \implies T_f = 990{}^\circ F
\end{equation}

Considering a constant heat flux along the tube wall, we can use the Lyon-Martinelli correlation:

\begin{equation}
Nu = 7 + 0.025Pe^{0.8}
\end{equation}

$Pe = Re.Pr = 0$ in our case, so, $Nu = \frac{hD_e}{k} = 7$.

From this, we can obtain $h = \frac{7k}{D_e} = \frac{7 * 50}{1/12} = 4200\ Btu.h^{-1}.ft^{-2}.{}^\circ F^{-1}$.

We can now calculate the heat flux, using:

\begin{equation}
q''_w = h(T_f - T_w)
\end{equation}

Consequently, $q''_w = - 4200 * (100) = -\text{\num{4.2e6}}\ Btu.h^{-1}.ft^{-2}$

This value for the heat flux changes with the bulk temperature chosen.

\section{[10-7] - Fluid-fueled reactor}
\label{prob102}


\subsection{Problem}
\textit{A fast fluid-fueled reactor uses uranium metal dissolved in liquid bismuth wit ha $U^{235}$ density of \num{1e20} $n.cm^{-3}$. The fuel thermal conductivity is $9\ Btu.h^{-1}.ft^{-1}.{}^\circ F^{-1}$. The fission cross section is 5b. The core can be approximated by a cylinder 3 ft in diameter. At a particular plane in the core, the neutron flux was flattened to \num{1e13} and the fuel bulk temperature is $782.5{}^\circ F$. If the fuel flow is assumed to be laminar, find for that plane (a) the wall temperature in case the walls were adiabiatic, (b) the centerline temperature for the preceding case, and (c) the percent heat generated that must be removed if, for structural reasons, the wall temperature should not exceed $700{}^\circ F$.}

\subsection{Solution}


In order to compute the wall temperature $T_w$, we can use Equation 10-22 of the book~\cite{book01}.

\begin{equation}
T_w = T_f + \frac{q'''r_0^2}{k}\frac{11F-8}{48}
\end{equation}

In the case of an adiabatic wall, $F = 1$. The only unknown in this equation is the volumetric thermal source strength, $q'''$. It can be obtained using the equation:

\begin{equation}
q''' = G \sigma_f N \phi g = 180 * \text{\num{5e-24}} * \text{\num{1e20}} * \text{\num{1e13}} * \text{\num{1.5477e-8}} = 13929\ Btu.h^{-1}.ft^{-3}
\end{equation}

$T_w$ is thus $1000{}^\circ F$.

Using Figure 10-9, we can obtain that:

\begin{equation}
\frac{T_w - T_c}{q'''r_0^2/2k_f} = 0.25
\end{equation}

Consequently, we can obtain $T_c = 565{}^\circ F$.
