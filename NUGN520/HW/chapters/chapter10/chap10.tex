%
% File: chap01.tex
%
\let\textcircled=\pgftextcircled
\chapter{Heat transfer and fluid flow, nonmetallic coolants}
\label{chap:intro}

\initial{S}everal exercises from the book written by M. M. El Wakil~\cite{book01} are tackled in this homework. The problems in this section relate to the tenth chapter of the book, covering the subject of heat transfer and fluid flow, for metallic coolants.

%=======
\section{[10-1] - Turbulent coolant tube}
\label{prob101}


\subsection{Problem}
\textit{A hypothetical liquid metal has $Pr = 0$ and $k = 50\ Btu.h^{-1}.ft^{-1}.{}^\circ F^{-1}$. It flows through a 1-in.-diameter tube with a Reynolds number of $1000000$. The tube-wall temperature is $1000{}^\circ F$. The temperature of the fluid halfway between the wall and the centerline is $900{}^\circ F$. Find (a) the temperature at the centerline of the tube, (b) the bulk temperature of the fluid and (c) the theat flux in $Btu.h^{-1}.ft^{-2}$.}

\subsection{Solution}

Martinelli provides solutions for the temperature gradient in pipe flow for various Reynolds and Prandtl number (and conveniently, $Re = 10^6$ and $Pr = 0$ is one of these solutions), seen in Figure 9-3 of the book~\cite{book01}.

We thus obtain, for $Pr=0$ and $Re = 10^6$:

\begin{equation}
\frac{T_w - T(y)}{T_w-T_c} = \frac{y}{r_0}
\end{equation}

Knowing $T_w = 1000{}^\circ F$ and also knowing $T(y=r_0/2) = 900{}^\circ F$, we can now easily solve the next equation for $T_c$:

\begin{equation}
\frac{T_w - T(y=r_0/2)}{T_w-T_c} = \frac{1}{2}
\end{equation}

\begin{equation}
T_c = 2*(900+500-1000) = 800{}^\circ F
\end{equation}

This is expected, we have a linear profile. In this case, the bulk temperature can be taken to be the average temperature in the channel, thus $T_f = 900{}^\circ F$. The book also states that (p. 236) "in highly turbulent flow, the temrature profile is fairly flat  over much of the cross-section, and the bulk temperature is taken as equal to the temperature at the center of the channel", so $T_f$ could also be taken as $800{}^\circ F$.

Considering a constant heat flux along the tube wall, we can use the Lyon-Martinelli correlation:

\begin{equation}
Nu = 7 + 0.025Pe^{0.8}
\end{equation}

$Pe = Re.Pr = 0$ in our case, so, $Nu = \frac{hD_e}{k} = 7$.

From this, we can obtain $h = \frac{7k}{D_e} = \frac{7 * 50}{1/12} = 4200\ Btu.h^{-1}.ft^{-2}.{}^\circ F^{-1}$.

We can now calculate the heat flux, using:

\begin{equation}
q''_w = h(T_f - T_w)
\end{equation}

Consequently, $q''_w = - 4200 * (100) = -\text{\num{4.2e6}}\ Btu.h^{-1}.ft^{-2}$

\section{[10-7] - Fluid-fueled reactor}
\label{prob102}


\subsection{Problem}
\textit{A fast fluid-fueled reactor uses uranium metal dissolved in liquid bismuth wit ha $U^{235}$ density of \num{1e20} $n.cm^{-3}$. The fuel thermal conductivity is $9\ Btu.h^{-1}.ft^{-1}.{}^\circ F^{-1}$. The fission cross section is 5b. The core can be approximated by a cylinder 3 ft in diameter. At a particular plane in the core, the neutron flux was flattened to \num{1e13} and the fuel bulk temperature is $782.5{}^\circ F$. If the fuel flow is assumed to be laminar, find for that plane (a) the wall temperature in case the walls were adiabiatic, (b) the centerline temperature for the preceding case, and (c) the percent heat generated that must be removed if, for structural reasons, the wall temperature should not exceed $700{}^\circ F$.}

\subsection{Solution}


In order to compute the wall temperature $T_w$, we can use Equation 10-22 of the book~\cite{book01}.

\begin{equation}
T_w = T_f + \frac{q'''r_0^2}{k}\frac{11F-8}{48}
\end{equation}

In the case of an adiabatic wall, $F = 1$. The only unknown in this equation is the volumetric thermal source strength, $q'''$. It can be obtained using the equation:

\begin{equation}
q''' = G \sigma_f N \phi g = 180 * \text{\num{5e-24}} * \text{\num{1e20}} * \text{\num{1e13}} * \text{\num{1.5477e-8}} = 13929\ Btu.h^{-1}.ft^{-3}
\end{equation}

$T_w$ is thus $1653{}^\circ F$.

Using Figure 10-9, we can obtain that:

\begin{equation}
\frac{T_w - T_c}{q'''r_0^2/2k_f} = 0.25
\end{equation}

Consequently, we can obtain $T_c = -88{}^\circ F$ 












































We can use Equation 5.51a from the book~\cite{book01}, which states, for a cylindrical fuel element with cladding and coolant:

\begin{equation}
T_{f,m} - T_b = \frac{q''' r_f^2}{4 k_f} + \frac{q'''r_f^2}{2} \left( \frac{1}{k_g} \ln \left( \frac{r_f + r_w}{r_f} \right) + \frac{1}{h(r_f + r_w)} \right)
\end{equation}

Where:

\begin{conditions}
T_{f,m} & Maximum temperature in the fuel \\
T_b & Coolant bulk temperature \\
r_f & Fuel radius position \\
r_w & Wall radius position \\
k_g & Aged thermal conductivity of graphite
\end{conditions}

We can reorganize the terms to express $h$. Doing so gives us:


\begin{equation}
h = \frac{1}{ \left( \left( 5000 - 1000 - \frac{\text{\num{2e6}} * (0.25/12)^2}{4 * 1.1} \right) * \frac{2}{\text{\num{2e6}} * (0.25/12)^2} - \left( \frac{1}{35} \ln \left( \frac{0.35/12}{0.25/12} \right) \right) \right) * \frac{0.35}{12}}
\end{equation}

And so, $h = 3.9\ Btu.h^{-1}.ft^{-2}.{}^\circ F^{-1}$.

We can now use the Dittus-Boelter equation, $Nu = 0.023Re^{0.8}Pr^{0.4}$, to compute the equivalent diameter needed. The Nitrogen bulk properties are used in this correlation, except for the fact that the free-stream temperature must be replaced by the adiabatic wall temperature $T_{fa}$.

This equation translates to:

\begin{equation}
\frac{hD_e}{k} = 0.023 \left( \frac{D_e v \rho}{\mu} \right)^{0.8} \left( \frac{c_p \mu}{k} \right)^{0.4} 
\end{equation}

This gives:


\begin{equation}\label{eq910}
D_e = \left( 0.023 \left( \frac{k}{h} \frac{v \rho}{\mu} \right)^{0.8} \left( \frac{c_p \mu}{k} \right)^{0.4} \right)^5
\end{equation}

The adiabiatic wall temperature can be obtained using the Mach number $M$ and the recovery factor $F_R$. Indeed, Equation 9-41 of the book~\cite{book01} gives:


\begin{equation}
T_{fa} = F_R (T_{fs} - T_f) + T_f
\end{equation}

Equation 9-40 gives:

\begin{equation}
T_{fs} = T_f \left( 1 + \frac{\gamma-1}{2}M \right)
\end{equation}

And the Mach number is given by Equation 9-37:

\begin{equation}
M = \frac{v}{\sqrt{\gamma R g_c T_f}}
\end{equation}

Where:

\begin{conditions}
\gamma & $1.4$ for Nitrogen \\
R & $54.99\ ft.lb_f/lb_m$ \\
T_f & $1459.7\ {}^\circ R$ \\
g_c & \num{4.17e8} \\
v & \num{1.35e6} fph
\end{conditions}

We thus obtain $M = 0.197$, $T_{fs} = 1007.8\ {}^\circ F$ and, using $F_R = 0.89$, $T_{fa} = 1006.9\ {}^\circ F$.

Using data from tables of Nitrogen at 6 atm and $1006.9{}^\circ F$, we get values extremely similar to the ones at $1000{}^\circ F$:

\begin{conditions}
k & $32.3\ Btu.h^{-1}.ft^{-1}.{}^\circ F^{-1}$\\
v & $375\ fps = \text{\num{1.35e6}}\ fph$\\
\rho & $0.16\ lb.ft^{-3} $\\
\mu & $8.76\ lb.ft^{-1}.h^{-1}$\\
c_p & $0.27\ Btu.lb^{-1}.{}^\circ F^{-1}$\\
h & $3.9\ Btu.h^{-1}.ft^{-2}.{}^\circ F^{-1}$
\end{conditions}

This data, when plugged into Equation~\ref{eq910} gives an erroneous number for the equivalent diameter, with the very high velocity causing the diameter to be way too high to get to the given heat transfer coefficient.

\section{[9-6] - W'/q}
\label{prob92}


\subsection{Problem}
\textit{Ligth liquid water is used as a reactor coolant. In a particular channel, the average bulk temperature is $300{}^\circ F$. Determine the percent change in $W'/q$ if it were to be used in the same channel, with the same mass-flow rate, the same mean temperature between cladding and coolant, but at average bulk temperature of $200{}^\circ F$ and $400{}^\circ F$. Assume saturated conditions in all cases.}

\subsection{Solution}

Equation 9.9 from the book~\cite{book01} can be used. It states:


\begin{equation}
\frac{W'}{q} = 7.07*10^{-14} \left( \frac{1}{D_e^{0.2}} \right) \left( \frac{v^{2.8}}{h\Delta T_m} \right) \rho^{0.8} \mu^{0.2} = \alpha \rho^{0.8} \mu^{0.2}
\end{equation}

$\alpha$ can be taken as a constant given the data. The only variable is the change in bulk temperature, implying a change in density and viscosity. The table from the appendix gives the following data:

\begin{conditions}
\rho(200{}^\circ F) & $60.1\ lb.ft^{-3}$ \\
\rho(300{}^\circ F) & $57.3\ lb.ft^{-3}$\\
\rho(400{}^\circ F) & $53.6\ lb.ft^{-3}$\\
\mu(200{}^\circ F) & $0.74\ lb.ft^{-1}.h^{-1}$\\
\mu(300{}^\circ F) & $0.45\ lb.ft^{-1}.h^{-1}$\\
\mu(400{}^\circ F) & $0.33\ lb.ft^{-1}.h^{-1}$
\end{conditions}

So, we get :

\begin{equation}
\frac{W'}{q}\bigg\rvert_{T_b = 300{}^\circ F} = 21.7 \alpha
\end{equation}


\begin{equation}
\frac{W'}{q}\bigg\rvert_{T_b = 200{}^\circ F} = 24.9 \alpha = 11.1\% \frac{W'}{q}\bigg\rvert_{T_b = 300{}^\circ F}
\end{equation}


\begin{equation}
\frac{W'}{q}\bigg\rvert_{T_b = 400{}^\circ F} = 19.4 \alpha = -10.6\% \frac{W'}{q}\bigg\rvert_{T_b = 300{}^\circ F}
\end{equation}

\section{[J1] - Temperature distribution}
\label{prob93}


\subsection{Problem}
\textit{For pressure driven laminar flow between parallel plates of separation $h$, the velocity components are $u = U(1-y^2/h^2)$, $v = w = 0$, where $U$ is the centerline velocity. Similarly to the example done in class, use an energy equation to find the temperature distribution $T(y)$ for a constant wall temperature $T_w$.}


\subsection{Solution}
We can write the energy equation:

\begin{equation}
\rho c_p \left( \frac{\partial T}{\partial t} + \vec{v}.\nabla T \right) = k \nabla^2 T + \mu \Phi + \dot{q}
\end{equation}

In a steady state, with constant heat generation and no dissipation terms, we can write:


\begin{equation}
\rho c_p \left( \vec{v}.\nabla T \right) = k \nabla^2 T
\end{equation}

In 2-D cartesian coordinates:

\begin{equation}
\rho c_p \left( u\frac{\partial T}{\partial x} + v\frac{\partial T}{\partial y} \right) = k \left( \frac{\partial^2 T}{\partial x^2} + \frac{\partial^2 T}{\partial y^2} \right)
\end{equation}

Noting that $\alpha = \frac{k}{\rho c_p}$, and that $v = 0$, we can write:


\begin{equation}
\frac{u}{\alpha} \frac{\partial T}{\partial x} = \frac{\partial^2 T}{\partial x^2} + \frac{\partial^2 T}{\partial y^2}
\end{equation}

We can assume that the rate of change of temperature in the x-direction will be negligible compared to the rate of change in temperature in the y-direction, thus, $\frac{\partial^2 T}{\partial x^2} << \frac{\partial^2 T}{\partial y^2}$. We can now sustitute the velocity distribution $u(y)$:


\begin{equation}
\frac{U}{\alpha} \left( 1 - \frac{y^2}{h^2} \right) \frac{\partial T}{\partial x} = \frac{\partial^2 T}{\partial y^2}
\end{equation}

Integration with respect to y, we obtain:

\begin{equation}
\frac{\partial T}{\partial y} = \frac{U}{\alpha} \frac{\partial T}{\partial x} \left( y - \frac{y^3}{3h^2} \right) + C_1
\end{equation}

Knowing the symmetry boundary condition $\frac{\partial T}{\partial y}\bigg\rvert_{y = 0} = 0$, we can set $C_1 = 0$. Integrating a second time:


\begin{equation}
T(y) = \frac{U}{2\alpha} \frac{\partial T}{\partial x} \left( y^2 - \frac{y^4}{6h^2} \right) + C_2
\end{equation}

Knowing the boundary condition $T(y = 0) = T_m$, we can set $C_2 = T_m$. However, this implies knowing $T_m$. Another boundary condition states that $T(y = h/2) = T_w$, $T_w$ being the temperature at the plate surface. In that case, $C_2 = T_m = T_w - \frac{23}{192}\frac{U}{\alpha}\frac{\partial T}{\partial x} h^2$.

And so,


\begin{equation}
T(y) = \frac{U}{2\alpha} \frac{\partial T}{\partial x} \left( y^2 - \frac{y^4}{6h^2} \right) + T_w - \frac{23}{192}\frac{U}{\alpha}\frac{\partial T}{\partial x} h^2
\end{equation}


\section{[J2] - Triangular pipe}
\label{prob94}


\subsection{Problem}
\textit{Consider steady laminar fluid flow in a fixed duct of equilateral triangular cross section. Take the side length as $s$ and the center of the triangle at the origin, with one side parallel to the $x$ axis. (a) Write the equations of the line segments defining the sides of the triangle. (b) Using from (a) a product form for the velocity of the fluid, develop a solution of the Navier-Stokes equation. Gravity may be ignored.}

\subsection{Solution}

The equation defining the segments of the triangle are:


\begin{equation}
y = -\frac{s}{2\sqrt{3}}
\end{equation}

\begin{equation}
y = x\sqrt{3} + \frac{s}{\sqrt{3}}
\end{equation}

\begin{equation}
y = -x\sqrt{3} + \frac{s}{\sqrt{3}}
\end{equation}

We can then use the three equations above to guess a solution that automatically satisfies all three zero velocity boundary conditions at the three wall. The solution would thus be of the form:

\begin{equation}
v_z = C_1 S(x,y) = C_1 \left( y + \frac{s}{2\sqrt{3}} \right) \left( y + x\sqrt{3} - \frac{s}{\sqrt{3}} \right) \left( y - x\sqrt{3} - \frac{s}{\sqrt{3}} \right)
\end{equation}

The Navier-Stokes equation can be written:

\begin{equation}
\mu \left( \frac{\partial^2 v_z}{\partial x^2} + \frac{\partial^2 v_z}{\partial y^2} \right) = \frac{\partial p}{\partial z}
\end{equation}

And so:

\begin{equation}
\frac{\partial^2 v_z}{\partial x^2} = C_1 (-6y - s\sqrt{3})
\end{equation}

\begin{equation}
\frac{\partial^2 v_z}{\partial y^2} = C_1 (6y - s\sqrt{3})
\end{equation}


\begin{equation}
\frac{\partial^2 v_z}{\partial x^2} + \frac{\partial^2 v_z}{\partial y^2} = - 2 C_1 s\sqrt{3}
\end{equation}

Substituting this equation into the Navier-Stokes equation:


\begin{equation}
- 2 C_1 s\sqrt{3} = \frac{1}{\mu}\frac{\partial p}{\partial z} \implies C_1 = \frac{1}{2\mu s\sqrt{3}}\frac{\partial p}{\partial z}
\end{equation}

And consequently,


\begin{equation}
v_z = \frac{1}{2\mu s\sqrt{3}}\frac{\partial p}{\partial z} \left( y + \frac{s}{2\sqrt{3}} \right) \left( y + x\sqrt{3} - \frac{s}{\sqrt{3}} \right) \left( y - x\sqrt{3} - \frac{s}{\sqrt{3}} \right)
\end{equation}

\section{[J3] - Scaling of the Navier-Stokes equation}
\label{prob95}


\subsection{Problem}
\textit{Scaling for the Navier-Stokes (NS) equation.  Consider the NS equation with gravity neglected, $\rho \frac{d\vec{V}}{dt} = - \nabla p + \mu \nabla^2 \vec{V}$. Suppose that $p(\vec{x}, t)$ and $\vec{V}(\vec{x}, t)$ solve these equations. There are scaling relationships, $\vec{V_{\tau}} = \tau^{\alpha} \vec{V} (\tau^{\beta} \vec{x}, \tau^{\gamma} t)$ and $p_{\tau} = \tau^{\delta} p (\tau^{\beta} \vec{x}, \tau^{\gamma} t)$ for any positive $\tau$, such that $p_{\tau}$ and $\vec{V_{\tau}}$ also solve the NS equation. (a) What are the relations between the exponents $\alpha, \beta, \gamma, \text{ and } \delta$ for this scaling to hold? (b) The equation of continuity also holds, $\nabla \cdot \vec{V_{\tau}} = 0$. Does this impose any further condition on the four exponents? Why or why not?}

\subsection{Solution}

The Navier-Stokes equation, with gravity neglected, is:

\begin{equation}
\rho \left( \frac{\partial \vec{v}}{\partial t} + \vec{v}.\nabla \vec{v} \right) = -\nabla p + \mu \nabla^2 \vec{v}
\end{equation}

We have various scaling propositions:

\begin{conditions}
\vec{v_{\tau}} & $\tau^{\alpha} \vec{v}$ \\
p_{\tau} & $\tau^{\delta} p$ \\
\vec{x_{\tau}} & $\tau^{\beta} \vec{x}$ \\
t_{\tau} & $\tau^{\gamma} t$ \\
\nabla_{\tau} & $\tau^{-\beta}\nabla$
\end{conditions}

We can now rewrite the Navier-Stokes equation:

\begin{equation}
\rho \tau^{\gamma - \alpha} \left( \frac{\partial \vec{v_{\tau}}}{\partial t_{\tau}} \right) + \rho \tau^{\beta - 2\alpha} \left( \vec{v_{\tau}}.\nabla_{\tau} \vec{v_{\tau}} \right) = - \tau^{\beta - \delta} \nabla_{\tau} p_{\tau} + \tau^{2\beta - \alpha} \mu \nabla_{\tau}^2 \vec{v_{\tau}}
\end{equation}

Multiplying both sides by $\frac{\tau^{2\alpha - \beta}}{\rho}$, we obtain the dimensionless:

\begin{equation}
\tau^{\alpha + \gamma - \beta} \left( \frac{\partial \vec{v_{\tau}}}{\partial t_{\tau}} \right) + \vec{v_{\tau}}.\nabla_{\tau} \vec{v_{\tau}} = - \frac{1}{\rho}\tau^{2\alpha - \delta} \nabla_{\tau} p_{\tau} + \tau^{\alpha + \beta} \frac{\mu}{\rho} \nabla_{\tau}^2 \vec{v_{\tau}}
\end{equation}


Knowing that $\vec{v_{\tau}}$ and $p_{\tau}$ verifies the Navier-Stokes equation, we can deduce that:


\begin{equation*}
\begin{cases} \tau^{\alpha + \gamma - \beta} & = 1 \\ \tau^{2\alpha - \delta} & = 1 \\ \tau^{\alpha + \beta} & = 1 \end{cases} \quad
\implies \begin{cases} (\alpha + \gamma - \beta)*\ln(\tau) & = 1 \\ (2\alpha - \delta)*\ln(\tau) & = 1 \\ (\alpha + \beta)*\ln(\tau) & = 1 \end{cases} \quad
\implies \begin{cases} \tau = 1 & \text{ or:} \\ \alpha + \gamma - \beta & = 0 \\ 2\alpha - \delta & = 0 \\ \alpha + \beta & = 0 \end{cases} \quad
\implies \begin{cases} \gamma & = -2\alpha \\ \delta & = 2\alpha  \\ \beta & = - \alpha \end{cases} \quad
\end{equation*}


We also know that the continuity equation (for an incompressible flow in this case) holds:

\begin{equation}
\nabla \cdot \vec{v_{\tau}} = \nabla \cdot \vec{v} = 0
\end{equation}

Considering only the x-direction:

\begin{equation}
\nabla \cdot \vec{v_{\tau}} = \tau^{-\alpha}\tau^{-\beta}\frac{\partial \vec{v_{x, \tau}}}{\partial x_{\tau}} = 0
\end{equation}

The term $\tau{-\alpha - \beta}$ simplifies out of the equation. Consequently, we haven't gained any insight on the scaling parameters.
