%
% File: abstract.tex
% Author: V?ctor Bre?a-Medina
% Description: Contains the text for thesis abstract
%
% UoB guidelines:
%
% Each copy must include an abstract or summary of the dissertation in not
% more than 300 words, on one side of A4, which should be single-spaced in a
% font size in the range 10 to 12. If the dissertation is in a language other
% than English, an abstract in that language and an abstract in English must
% be included.

\chapter*{Executive Summary}
\begin{SingleSpace}
\initial{V}arious risk and reliability analysis methods were applied to the ASTRID nuclear reactor design. The design demonstrated remarkable safety features, with exceedingly low chances of accidents causing the release of radioactivity in the environment. A more complete analysis could reveal potential cost savings actions while keeping the system safety extremely high.

This improved safety comes at the cost of a slightly downgraded reliability, with the reactor being shutdown preemptively in numerous cases as part of the defense system. It is recommended to consider a coupled system for electricity generation to cover for most of the plant downtime and drastically increase cost effectiveness.

One of the point of interest seen in this analysis that would merit a more in depth analysis is the loss of offsite power, forcing the use of numerous, historically often unreliable, backup generators and shutting down the plant due to lack of powered sensors in the core. Alternative electricity generation solution should be studied.

This analysis should be repeated with greater details once the design is more advanced, in order to confirm the findings of this document and potentially make informed cost-saving modifications.

The numbers and design choices considered in this study are based on the author's engineering background and may not reflect the reality. While the qualitative conclusions can be taken with a reasonable degree of confidence, the values given in this document should not be reused for an official risk and reliability analysis. Due to time constraints, several subsystems and paths to failure have been ignored in this document. Those intentional omission may prove to be more limitating than the ones explicited.
\end{SingleSpace}
\clearpage

\chapter*{Abstract}
\begin{SingleSpace}
\initial{T}he importance of understanding, assessing, communicating, and making decisions based in part upon risk, reliability, robustness, and uncertainty is rapidly increasing in a variety of industries (e.g.: petroleum, electric power production, etc.) and has been a focus of some industries for many decades (e.g.: nuclear power, aerospace, automotive, etc). This project aims at applying a number of different risk and reliability analysis methods to gain insight on a particular complex system.

One of the leading industry in the risk and reliability engineering field is the nuclear power industry. Nuclear power is coming to a turning point, which will likely decide its future. Second generation reactors designs, developed in the 50s and 60s, are used today to generate most of the world's nuclear energy. Accidents like Chernobyl and Fukushima have led to heavy criticism of the nuclear industry by a large number of lay people.

Several third generation reactor designs are being built today to replace the world aging nuclear fleet, but they are already under criticism, being considered too risky. The fourth generation reactor design developments are still underway, and have the ability to change lay people's view on this source of energy. This can be accomplished only if the risks are analyzed and taken into account to the best of our abilities, and if these studies' results are communicated efficiently to the unforgiving public opinion.

In that regard, a fourth generation nuclear reactor prototype, the Advanced Sodium Technological Reactor for Industrial Demonstration (ASTRID), is under development by the CEA in France. Its goal is to demonstrate the feasability of such designs, from a technical and economical standpoint. A particularly interesting point in light of this study is that this Sodium-cooled design presents some obvious risks, sodium-water and sodium-air interactions, and an interesting history.
\end{SingleSpace}
\clearpage
