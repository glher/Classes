%
% File: abstract.tex
% Author: V?ctor Bre?a-Medina
% Description: Contains the text for thesis abstract
%
% UoB guidelines:
%
% Each copy must include an abstract or summary of the dissertation in not
% more than 300 words, on one side of A4, which should be single-spaced in a
% font size in the range 10 to 12. If the dissertation is in a language other
% than English, an abstract in that language and an abstract in English must
% be included.

\chapter*{Abstract}
\begin{SingleSpace}
\initial{T}he importance of understanding, assessing, communicating, and making decisions based in part upon risk, reliability, robustness, and uncertainty is rapidly increasing in a variety of industries (e.g.: petroleum, electric power production, etc.) and has been a focus of some industries for many decades (e.g.: nuclear power, aerospace, automotive, etc). This project aims at applying a number of different risk and reliability analysis methods to gain insight on a particular complex system.

One of the leading industry in the risk and reliability engineering field is the nuclear power industry. Nuclear power is coming to a turning point, which will likely decide its future. Second generation reactors designs, developed in the 50s and 60s, are used today to generate most of the world's nuclear energy. Accidents like Chernobyl and Fukushima have led to heavy criticism of the nuclear industry by a large number of lay people.

Several third generation reactor designs are being built today to replace the world aging nuclear fleet, but they are already under criticism, being considered too risky. The fourth generation reactor design developments are still underway, and have the ability to change lay people's view on this source of energy. This can be accomplished only if the risks are analyzed and taken into account to the best of our abilities, and if these studies' results are communicated efficiently to the unforgiving public opinion.

In that regard, a fourth generation nuclear reactor prototype, the Advanced Sodium Technological Reactor for Industrial Demonstration (ASTRID), is under development by the CEA in France. Its goal is to demonstrate the feasability of such designs, from a technical and economical standpoint. A particularly interesting point in light of this study is that this Sodium-cooled design presents some obvious risks, sodium-water and sodium-air interactions, and an interesting history.
\end{SingleSpace}
\clearpage
