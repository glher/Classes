\typeout{IJPHM template}
\typeout{Template version updated November 25, 2013}

% Please send questions regarding the Latex templates to Indranil Roychoudhury (indranil.roychoudhury@nasa.gov) or Matthew Daigle (matthew.j.daigle@nasa.gov).
 

\documentclass[IJPHM, 2017, 29]{PHMSociety}

%Declare Packages
\usepackage{graphicx}
\usepackage{amsmath}
\usepackage{kbordermatrix}
\usepackage{mathrsfs}
\usepackage{subcaption}
\usepackage{multirow}
\usepackage{siunitx}
\usepackage{pbox}
\usepackage [autostyle, english = american]{csquotes}
%\usepackage{draftwatermark}
\usepackage{booktabs}% http://ctan.org/pkg/booktabs
\usepackage[hyphens,spaces,obeyspaces]{url}
\newcommand{\tabitem}{~~\llap{\textbullet}~~}
%\SetWatermarkText{Draft-5}
%\SetWatermarkScale{1}
\MakeOuterQuote{"}
\hyphenation{Bay-es-ian
             data-base
             data-bases
             Sw-ain}
\newenvironment{conditions}
  {\par\vspace{\abovedisplayskip}\noindent\begin{tabular}{>{$}l<{$} @{${}={}$} l}}
  {\end{tabular}\par\vspace{\belowdisplayskip}}

\begin{document}
\renewcommand{\kbldelim}{[}% Left delimiter
\renewcommand{\kbrdelim}{]}% Right delimiter

% Paper Title
\title{Sustainability of Nuclear Power in an evolving energy market}

% Authors List
\author{%			
	Guillaume L'Her\authorNumber{1,\thanks{Corresponding Author}}
}

% Author Affiliations
\address{% This is a tabular environment so each affiliation needs to be separated by "\\" or "\tabularnewline"
	\affiliation{{1}}{Colorado School of Mines, Golden, CO, 80401, USA}{ %add emails
		{\email{glher@mines.edu}} 
		} % emails input
}

% Create the title
\maketitle
\pagestyle{fancy}
\thispagestyle{plain}


% PHM Society Distribution License Information, provide first author's name "FirstName LastName"
\phmLicenseFootnote{Guillaume L'Her}

% Abstract
\begin{abstract}%   %NOTE: Deleting the percentage after "{abstract}" may be lead to an extra leading space in the first line of the abstract, and this should be prevented.
The energy market is evolving rapidly, with the development of fully renewable energy.
\end{abstract}



\section{Introduction}
\label{sec:intro}
The energy market landscape is changing. Several factors can be considered to explain this phenomenon, notably the development of new technology and the growing importance of public opinion. The research and population interests in the energy generation domain is relatively recent and exacerbated by indirect drivers such as climate change, fear and fossil supply uncertainty, among others. Wind energy installed capacity has already surpassed that of Nuclear, and solar energy is on its way to do the same within the next couple years. Of course, when it comes to wind or solar energy, installed capacity is much higher than actual electricity production.

In the meantime, the nuclear industry was severely hit by a loss of public trust following the accident at the Fukushima-Daichi nuclear power plant in March of 2011. This sparked political agendas to push the energy debate in the center of their campaign platforms in several influent countries. Public perception is that no progress is seemingly made in the disposal of waste products. The nuclear industry now faces a hostile public opinion often pushed by social media and conventional medias. Combined with excitement about solar and wind energy and their lowering costs, the long-term place of nuclear in the energy landscape can be considered in jeopardy.

This paper proposes a method to find the energy mix with the highest chance of success based on several factors (public, politics, economics, technology) and assume an organic evolution of the market toward this scenario. The future of nuclear power in the evolving energy mix will consequently be computed.

Section~\ref{sec:bkgd} presents the background necessary for the present paper. Recent advances and inherent issues of truly renewable energy source (solar energy, wind energy, hydroelectric energy) are discussed and compared to Nuclear energy. Then, the methodology used in this paper is introduced in section~\ref{sec:meth}. Several scenarios that will be compared are given. The results are obtained in section~\ref{sec:study}. Finally, section~\ref{sec:fut} debates the work produced in this paper and the potential future improvement on the method, while section~\ref{sec:conclusion} gives the study conclusion.


\section{Background}
\label{sec:bkgd}

\subsection{Solar energy market}

The solar energy, and more specifically Photovoltaic (PV) technology, is expanding at a near-exponential rate. In the year 2016, installed solar energy capacity expanded by close to 34\%, jumping from 227 GW to 303 GW~\citep{iea_solar}. The main contributor countries were China, India, and the United States. If the expansion rate continues at similar levels, the installed power capacity of solar energy will surpass the nuclear capacity in a couple of years, even though the actual electricity generation will be much lower, due to the inherent limitation of solar energy and its capacity factor of 24\% in a best case scenario in the US Southwest~\citep{solar_comp} as opposed to around 85\% for nuclear.

Solar energy may be showing its limits when it comes to efficiency and deployable technology. Indeed, Photovoltaic elements using Silicon dominate the solar energy market. Silicon's low cost, reasonable efficiency and widespread use makes it a high barrier (mostly due to cost) to enter the market for new, more efficient technology. Short of a consequential breakthrough, solar can be thought to have reached its individual cells potential.

Solar present several important advantages over other forms of energy. The main one is its versatility. It can be installed on the top of building, on various surfaces, and can be made to be form-fitting. Organic PV are also a potential huge additional market, as despite low efficiency, it can be processed on low cost substrates with standard coating processes. Moreover, solar energy benefits from an interesting coolness factor and has the full support of public opinion.

Solar energy production can easily be predicted a day in advance, with slight unpredictable variations due to cloud coverage. However, it is a diurnal technology, with a peak around noon. This can be problematic for two main reasons. The current electrical grid is not made to account for such variations and has next to no electricity storage ability, and the peak production time of solar energy happens during the least demanding time of day from an electricity point of view~\citep{solar_peak}.


\subsection{Wind energy market}

Wind energy presents several drawbacks compared to solar energy. It is a lot more erratic and unpredictable, and much more geographically limited than solar energy source. A wind turbine site can be down for several days in a row. Installed capacity is higher than that of nuclear and solar, but the capacity factor for wind powered generation has been lower than anticipated~\citep{wind_cf}, at around 20\%.

Wind powered energy source does not exhibits the same public support as solar energy, notably due to its perceived impact on the environment, both in terms of wildlife and landscape. Its development, in contrast with solar energy, has been underwhelming, below projections~\citep{wind_perf}.


\subsection{Hydroelectric energy market}

Hydroelectric energy is almost at capacity in most countries. However, it can be used as a storage unit for excess energy from other sources, with pumped storage technology. This demonstrates a non-negligible benefit of this power source: readily available power on very short notice. As such, until the advent of efficient battery storage in the grid, hydroelectric power is one of the few energy sources, with coal/gas to a lesser extent, to be able to respond to unexpected peak demand from the grid. 

The highest producing electricity plants in the world are hydroelectric dams, but small hydroelectric units are considered for development~\citep{hydro_small}.


\subsection{Review}

Amongst the main three renewable energy sources, solar, wind and hydroelectric, solar has consistently beaten installed capcity projections, while wind has not. Hydroelectric is almost at full capacity in most countries. The difficulty in integrating solar and wind to the grid is their difficult predictability, low capacicity factor and unhelpful peak hours. As long as these problems remain, the grid cannot be stable enough to expand the use of such energy sources, and a base load is required.

This generates the loss of renewable electricity when the demand is lower than production. Additionally, the renewable energy produced is sold at the lowest cost to the grid, due to the low demand. Efficient storage technology would be a game changer for the renewable sources, but until then, base loads are necessary.


\section{Methodology}
\label{sec:meth}

This section introduces the methodology that is developed in this paper. The goal of the methodology is to determine the most likely scenario according to a given crude simulation of the world, and to compute the most favorable importance of various factors in the success and sustainability of nuclear energy in the future.

\subsection{Factors}

Several contributing factors are introduced in Table~\ref{tab1}. These factors represents the different forces at play that drive the energy market today and in the future. It is important to note that while the factors can appear fully dependent, this may not be so. For example, the weights for the public opinion and for the environment can be thought to be tightly connected, but something good for the environment can be poorly received by the public, and the public can support something that can end up damaging the environment in the long term.

\setlength\extrarowheight{5pt}
\begin{table}[]
\centering
\caption{Factors and associated weights}
\label{tab1}
\begin{tabular}{|c|c|c|}
\hline
Bigramme & Factor ($F$)               & Weight ($W$) \\ \hline
PU & Public opinion           & 3      \\ \hline
TC & Technology               & 2      \\ \hline
EC & Economics                & 2      \\ \hline
EN & Environment              & 3      \\ \hline
PO & Political                & 2      \\ \hline
IN & Investment               & 1      \\ \hline
\end{tabular}
\end{table}

The various factors are weighted according to the importance one decides to give them. An example of weighting is given in this paper. A score $S$ between -5 and 5 is given to each factor depending on the scenario. Each score is attributed according to the reference scenario (current energy market). A score of -5 represents a negative impact of the scenario on the factor, 0 represents no impact of the scenario and a score of 5 represents a positive impact on the factor. The total scenario score $S_t$ is then computed using Equation~\ref{eq1}.

\begin{equation}
\label{eq1}
S_t = \sum_F {W_F S_F}
\end{equation}

The score associated with each factor is determined by expert judgement and should not be taken as granted. The factors will now be defined.

\subsubsection{Public opinion (PU)}

This factor represents the push that can be made by a technology if it is supported by public opinion. Public opinion brings in funding, advertisement and an available marketplace for proof of concept during the development stage of a technology or electric grid alternative. It can also impacts the political and economics factors. The higher the score, the more support the scenario would have from the public.

\subsubsection{Technology (TC)}

This factor represents the technological hurdles that would need to be overcome to deploy the scenario. It is linked to the investment factor.

\subsubsection{Economics (EC)}

This factor takes into account the operating and maintenance cost of a scenario.

\subsubsection{Environment (EN)}

This considers the impact of the scenario on the environment. The higher the score, the better it is for the environment. It is assumed that the world will naturally tend toward the better environment, all else being equal.

\subsubsection{Political (PO)}

This factor includes the public opinion but also the different industries lobbying. The higher the score, the more the politics align with the scenario and help make it happen.


\subsubsection{Investment (IN)}

This factor accounts for the inital cost of the development, and the cost to enter the market in the first place. For example, entering the market with a nuclear close cycle might be difficult as the upfront cost associated to prove the concept could be considered too high and risky.

Without funding, no project can expect to survive and get to the scaled prototype step. A set fund will be considered and divided into the different energy sources of the scenario according to needs. That is, if a scenario require both the development of fast spectrum nuclear reactor and PV technology, the score will be lower.

\subsection{Scenarios}
\label{sec:lfm}

Several representative scenarios are developed and will be compared to one another given an identical set of factors and weigths.

\setlength\extrarowheight{10pt}
\begin{table*}[]
\centering
\caption{Description of the considered energy mix scenarios}
\label{tab2}
\begin{tabular}{|c|l|}
\hline
Scenario & \multicolumn{1}{c|}{Description}                                                                                                            \\ \hline
Sc1      & \pbox{20cm}{This corresponds to the current scenario. The percentage of electricity generation are kept the same with increasing \\ demand.}                \\ \hline
Sc2      & \pbox{20cm}{In this scenario, 50\% of the nuclear power is replaced by renewable sources.}                                                               \\ \hline
Sc3      & \pbox{20cm}{In this scenario, the nuclear power plants are phased out and replaced by a nuclear closed cycle, the parts of renewable \\ and fossils stay the same} \\ \hline
Sc4      & \pbox{20cm}{Fossil fuel is replaced by renewable sources}                 \\ \hline
Sc5      & \pbox{20cm}{The nuclear plant are phased out and replaced by a nuclear closed cycle, while fossil fuel is replaced by renewable \\ sources}                 \\ \hline
Sc6      & \pbox{20cm}{The nuclear plants close, and are replaced by fossil fuel, while the renewable share stays the same.}                                        \\ \hline
Sc7      & \pbox{20cm}{The grid becomes 100\% renewable only}                                                                                                             \\ \hline
\end{tabular}
\end{table*}

The scenarios consider the current industrialized technology. It does not consider disruptive technology, such as nuclear fusion, easy and cheap electricity storage, or other such potential advances. The timeline of the grid modification proposed is not explicitely covered. An ASARA principle, As Soon As Reasonably Achievable, is assumed.

\subsection{Review}

Several factors, their associated importance (weigths) are defined. The impact of various scenarios on these factors is estimated and a scenario global score is obtained to compute its likelihood of being chosen.

\section{Case study}
\label{sec:study}

For each scenario, the factor scores $S_F$ are derived from expert judgement. The first scenario, Sc1, is discussed as an example.

In this scenario, half of the nuclear energy market share is replaced by renewables sources. The renewable sources in this scenario are mostly solar and wind, due to hydro being near capacity already. Such a move would receive a good support from the public, as it would be well perceived by the most vocal ecology people. A score of 4 is attributed to Public Opinion. The technology challenge to install that many renewable energy on the market and not have any grid insufficiency would be reasonably achievable. A score of -1 is given to the Technology factor. From an economy point of view, one could expect the operating cost to go down, because the maintenance and lower capacity factor would be shared between more entities. A score of 1 is given to the Economics factor. Politically, this would be perceived as a good move in many countries, due to growing ecological movements. Politics get a score of 3. Maybe counter-intuitively, the Environment would get a negative score, -1. This is due to the fact that in order to compensate for the loss of base load coming from nuclear and the low capacity factor from solar and wind powered energy, we would likely see an increase of fossil fuel to compensate, adding installed capacity to the grid in order to keep the same electricity production capacity. Finally, the investment needed is estimated to be slightly negative, at -1, due to the changes in the grid necessary to accomodate this higher number of uncertain energy sources.


\subsection{Results}

\begin{table}[]
\centering
\caption{Results}
\label{tab3}
\begin{tabular}{|c|c|c|c|c|c|c|c|}
\hline
Scenario & PU & TC & EC & PO & EN & IN & $S_t$ \\ \hline
Sc1      & 0  & 0  & 0  & 0  & 0  & 0  & 0     \\ \hline
Sc2      & 4  & -1 & 1  & 3  & -1 & -1 & 1.08  \\ \hline
Sc3      & 3  & -4 & 2  & 2  & 2  & -5 & 0.77  \\ \hline
Sc4      & 5  & 0  & 1  & 4  & 5  & -2 & 2.92  \\ \hline
Sc5      & 5  & -4 & 2  & 3  & 5  & -5 & 2.08  \\ \hline
Sc6      & -5 & 3  & 1  & -1 & -5 & 1  & -1.77 \\ \hline
Sc7      & 5  & -4 & -2 & 3  & 5  & -3 & 1.62  \\ \hline
\end{tabular}
\end{table}


The results, with the given weigths, show that the most likely scenario to pan out would be an energy mix between Gen III nuclear and renewables energy. Behind this comes the same scenario with a nuclear closed cycle (Gen IV) replacing the current nuclear power.

\section{Discussions and future work}
\label{sec:fut}
With current technology or slightly improved ones, but without considering breakthrough technology such as nuclear fusion or cheap, modular and scalable energy storage solutions, it can be seen that nuclear power is an integral part of the future energy market, either in its current form or as a closed fuel cycle with next generation power plants.

It is interesting to note that another nuclear accident would likely severely shift public opinion against the nuclear industry. In such a case, the scenario containing nuclear would see a lower score, and Scenarios 2 (renewable instead of nuclear) and 7 (fully renewable) would become more likely.

Along this reasoning line, various game changing events could be considered in the methodology. A game theory algorithm could then be applied to the different scenario to give enlightening results. This would be the subject of future work.


\section{Conclusion}
\label{sec:conclusion}

In its current state, nuclear power can be considered sustainable. A fully-closed nuclear fuel cycle, with waste burner, and breeder-burner reactors, could encounter a lot of difficulties to enter the energy market. Indeed, the high technological hurdles (notably material-wise) and high upfront cost might not make this technology worth the risk.

Any significant event such as a volcano eruption, nuclear accident, or a breakthrough discovery would impact the different scenarios and should be considered, along with a likelihood of happenstance.


\section*{Acknowledgment}
This work was supported by my precious free time. Any opinions or findings of this work are the responsibility of the author.




\section*{Nomenclature}

\begin{tabular}{ l  l }
    \textbf{F}           &Factor\\
    \textbf{$S_F$}           &Score of a factor\\
    \textbf{$W_F$}           &Weigth of a factor\\
    \textbf{$S_t$}           &Total scenario score\\
 \end{tabular}


\section*{Acronyms}

\begin{tabular}{ l  l }
    $PWR$           &Pressurized Water Reactor\\
    $PV$          &Photovoltaic
 \end{tabular}


\clearpage
% Bibliography
% ---------------------------------------------------------------------------------
\bibliographystyle{apacite}
\PHMbibliography{pfbn}
% ---------------------------------------------------------------------------------



\end{document}


