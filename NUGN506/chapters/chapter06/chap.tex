%
% File: chap01.tex
%
\let\textcircled=\pgftextcircled
\chapter{In-Core Fuel Management}
\label{chap:intro}

\initial{S}everal problems related to the in-core fuel management are tackled in this homework. The subjects of availability factor, capacity factor, cost of fuel and burnup are notably considered.

%=======
\section{Problem 6-1}
\label{prob61}


\subsection{Problem}
\textit{Prove that if a utility pays 1 mill/kWhe for disposal of used fuel, it is equivalent to \$253/kgU. Assume a burnup of 33,000 MWd/t and an average thermal efficiency of 32\%}

\subsection{Solution}

A utility pays 1 mill for every $kWh_e$. Considering the efficiency, this means that they pay 1 mill for every $3.125\ kWh_{th}$ they produce in disposal of used fuel. How much uranium is needed to produce $3.125\ kWh_{th} = 0.13\ kWd_{th} = 0.00013\ MWd_{th}$ ?

The plant produces 33,000 MWd per tons of uranium, so \num{3.94e-6} kg are needed to produce $1\ kWh_e$. Paying 1 mill for \num{3.94e-6} kgU means that the price of uranium is 253.8 \$/kgU


\section{Problem 6-2}
\label{prob62}


\subsection{Problem}
\textit{A utility is given the choice of paying 1.2 mill/kWhe for a used fuel disposal fee or \$300/kgU at 35,000 MWd/t burnup and 32\% efficiency. Which choice is more economical?}

\subsection{Solution}

If the utility picks the 1.2 mill/kWhe option, this translates to, as seen previously (with 35,000 instead of 33,000 MWd/t) : 323.1 \$/kgU. The utility should pick the flat fee.



\section{Problem 6-3}
\label{prob63}


\subsection{Problem}
\textit{The initial enrichment of uranium in LWRs is 3\% and the final (at discharge after 3 years in the core) is 0.8\%. Based on these numbers, calculate the average thermal neutron flux in a LWR.}

\subsection{Solution}

The burnup is 73.3\%, as it is the percentage of U-235 that has been consumed. We'll consider $\sigma_a = 685\ b$ for U-235, and a thermal group approximation.

We know that:

\begin{equation}
\frac{dN_5}{dt} = -\sigma_a * \phi * N_5
\end{equation}

Consequently:

\begin{equation}
\frac{N_5(1-0.266)}{3 * 86400 * 365.25} = \text{\num{685e-24}} * \phi * N_5
\end{equation}

Rearranging:


\begin{equation}
\phi = \frac{1-0.266}{3 * 86400 * 365.25 * \text{\num{685e-24}}}
\end{equation}

And so, the average flux is $\phi = \text{\num{1.13e13}}$. This is based on several consequential approximations: the absorption cross-section of U-235, the negligibility of the fast group in the reactor, and the burnup of only U-235.

\section{Problem 6-4}
\label{prob64}


\subsection{Problem}
\textit{What are the availability and capacity factor for a reactor that had the power history shown in Problem 6.4 from the book?}

\subsection{Solution}

The availability factor is $91.7\%$, as the plant was operational 11 months out of 12. The capacity factor is $67.5\%$.

\section{Problem 6-5}
\label{prob65}


\subsection{Problem}
\textit{Calculate the EFPD for the operating history shown in Problem 6.5 from the book}

\subsection{Solution}

$30.4375 * [2*1 + 1*0.75 + 1*0.5 + 4*1] = 220.7\ EFPD$

The operating history shows a EFPD of 220.7 days.


\section{Problem 6-6}
\label{prob66}


\subsection{Problem}
\textit{A nuclear power plant uses 87 tons of uranium in its core. The first core consists of three equal batches with enrichments of 2.5\%, 2.9\%, and 3.1\%. For subsequent cycles, the reactor uses batches with enrichment equal to 2.6\% and 3.2\% in alternate years. One-third of the core is discharged each year and is replaced by new fuel. Assume a 30-year lifetime and 0.2\% tails during the life of the plant. Calculate (a) the number of SWUs needed for the first core, the number of SWUs every year after the first cycle, and the total SWUs over the life of the plant; (b) the amount of U3O8 and UF6 needed for the first core, every year after, and the total over the life of the plant ; (c) assume the following prices and calculate the cost of the first core and the cost of the fuel per year for the life of the plant [U3O8: \$65/kg, conversion: \$7/kg with 0.5\% loss, SWU: \$105, fabrication and transportation: \$250/kg with 0.8\% loss]}

\subsection{Solution}

(a) We know that 29 tons of Uranium in the first core is enriched at 2.5\%:

The SWUs needed to get this is given by:

\begin{equation}
SWU = PV(x_p) + WV(x_w) - FV(x_f)
\end{equation}

We now need to calculate P, W, and F, knowing that $x_w = 0.002$, $x_f = 0.00711$ and $x_{p,1} = 0.025$.

P is given by $P = 87/3 = 29$ tons. Consequently:

\begin{equation}
F_1 = P\frac{x_{p,1} - x_w}{x_f - x_w} = 29\frac{0.025-0.002}{0.00711-0.002} = 130.5\ \text{tons}
\end{equation}

From that, we can deduce $W_1 = 130.5 - 29 = 101.5$ tons.

Finally, we can calculate the SWU needed for this particular batch of the first core:

\begin{equation}
SWU_1 = 29*v(0.025) + 101.5*v(0.002) - 130.5*v(0.00711) = 93.6 tSWU
\end{equation}

In the same way, we can calculate the SWU needed for the other two batches of the first core:

\begin{equation}
F_2 = P\frac{x_{p,2} - x_w}{x_f - x_w} = 29\frac{0.029-0.002}{0.00711-0.002} = 153.2\ \text{tons}
\end{equation}

\begin{equation}
SWU_2 = 29*v(0.029) + 124.2*v(0.002) - 153.2*v(0.00711) = 118.5\ tSWU
\end{equation}



\begin{equation}
F_3 = P\frac{x_{p,3} - x_w}{x_f - x_w} = 29\frac{0.031-0.002}{0.00711-0.002} = 164.6\ \text{tons}
\end{equation}

\begin{equation}
SWU_3 = 29*v(0.031) + 135.6*v(0.002) - 164.6*v(0.00711) = 131.3\ tSWU
\end{equation}

We can sum the three SWU obtained to get the total SWU needed for the first core, $343.4$ tSWU.

To simplify things, we'll consider that the cycle length is one year, with no significant downtime for refueling.

A 2.6\% batch reload would necessitate:

\begin{equation}
F_4 = P\frac{x_{p,4} - x_w}{x_f - x_w} = 29\frac{0.026-0.002}{0.00711-0.002} = 136.2\ \text{tons}
\end{equation}

\begin{equation}
SWU_4 = 29*v(0.026) + 107.2*v(0.002) - 136.2*v(0.00711) = 99.8\ tSWU
\end{equation}

A 3.2\% batch reload would necessitate:

\begin{equation}
F_5 = P\frac{x_{p,5} - x_w}{x_f - x_w} = 29\frac{0.032-0.002}{0.00711-0.002} = 170.3\ \text{tons}
\end{equation}

\begin{equation}
SWU_5 = 29*v(0.032) + 141.3*v(0.002) - 170.3*v(0.00711) = 137.7\ tSWU
\end{equation}

We'll have 15 reloads at 2.6\% and 14 reloads at 3.2\%, and so the total SWU over the life of the reactor is $343.3 + 14*137.7 + 15*99.8 = 3768.1$ tSWU.


(b) We can now calculate the amount of U3O8 and UF6 needed for the first core. We need $F = 448.3$ tons of feed UF6, as seen previously. This corresponds to 378.3 tons of U3O8 with a 0.5\% conversion loss. Over the life of the plant, we need $448.3 + 15 * 136.2 + 14 * 170.3 = 4875.5$ tons of UF6. This corresponds to $4113.7$ tons of U3O8.

(c) We can now calculate the costs. We now that we need 378.3 tons of U3O8, at \$65/kg, and 343.4 tSWU at \$105. So, in order to get the enriched fuel needed for the first core, we need \$24,589,500 for the U3O8 and \$36,057,000 for the enrichment. We have 87 tons of fuel to fabricate and transport for the first core, at \$250/kg with 0.8\% loss. To fabricate and transport 87 tons, we thus need to have 1.008 times the amount of fuel, which directly translates to an 0.8\% increase in the cost of the U3O8 and subsequent enrichment.

Finally:

\begin{equation}
1.008 * (\$24,589,500 + \$36,057,000) + 87,000 * \$250 = \$82,881,672
\end{equation}

The first core will cost a total of \$82,881,672.

For a 2.6\% reload, we need $1.008 * (99.8 * 1000 * \$105 + 136.2 * 1000 * \$65) = \$19,486,656$. For a 3.2\% reload, we need $1.008 * (137.7 * 1000 * 105 + 170.3 * 1000 * 65) = \$25,732,224$. In each case, we also need \$7,250,000 for fuel fabrication and transportation.

So, over 30 years:

\begin{equation}
\$82,881,672 + 15 * \$19,486,656 + 14 * \$25,732,224 + 29 * \$7,250,000 = \$945,682,648
\end{equation}

The fuel will cost a little less than a billion dollar over the 30-year lifespan of the plant. This comes at a cost of around \$600,000 per assembly assuming a common 157-assemblies 1000MWe reactor.


\section{Problem 6-7}
\label{prob67}


\subsection{Problem}
\textit{Is it possible for the availability and capacity factors to be equal? Explain your answer. Take into account the real operating conditions of a reactor.}

\subsection{Solution}

By definition, it is not possible. Indeed, a nuclear reactor needs to ramp up over the course of a week, which would impact the capacity factor but not the availability factor.


\section{Problem 6-8}
\label{prob68}


\subsection{Problem}
\textit{A nuclear power plant operated for a year with a 69\% capacity factor. What is the burnup of the fuel if the core contains 95 tons of uranium and has been designed to produce 1200 MWe with a 32\% efficiency? What will be the average burnup, over the life of the plant, if one-third of the core is replaced every year? Assume a 30-year life for the plant and a constant capacity factor and efficiency.}

\subsection{Solution}

The burnup is given by:

\begin{equation}
BU = \frac{P_0 CF T}{U_M} = \frac{3750 * 0.69 * 365.25}{95} = 9948.3\ MWd/t
\end{equation}

After 3 years, the plant reaches an equilibrium and the average burnup is constant, at 19896.6 MWd/t. When one batch reaches the maximum burnup before discharge after 3 years, the core contains one batch at 9948.3 MWd/t, one at 19896.6 MWd/t and one at 29844.9 MWd/t. Then, the rotation makes it so that the higher burnup batch is removed, a new one is introduced, and we again have the same batch burnup configuration after an additional year.
