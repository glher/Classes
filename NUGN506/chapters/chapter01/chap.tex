%
% File: chap01.tex
%
\let\textcircled=\pgftextcircled
\chapter{Introduction to the nucleus}
\label{chap:intro}

\initial{S}everal problems related to the nucleus are treated in this introductory homework. The first problem treats of binding energy for one nuclide, while the second tackles spontaneous fission and the third compares the release of energy for fission and fusion.

%=======
\section{Binding energy of Th-232}
\label{prob11}


\subsection{Problem}
\textit{Find the total binding energy and the binding energy per nucleon of Thorium-232 in MeV.}

\subsection{Solution}

We know that Thorium has an atomic number $Z=90$. It thus contains 90 protons. Thorium-232 is an isotope of Thorium, and contains $N = 232 - 90 = 142$ neutrons.

The mass of Thorium-232 is $232.985042\ amu$. The mass of a proton is $1.007277\ amu$ and the mass of a neutron is $1.008664\ amu$. Consequently, the "physical" mass of Thorium is:

\begin{equation}
M_{Th} = 90*1.007277 + 142*1.008664 = 233.885218\ amu
\end{equation}

We thus obtain a mass defect of:

\begin{equation}
\Delta M = 232.985042 - 233.885218 = - 0.900138\ amu
\end{equation}

And in kg:

\begin{equation}
\Delta M = - 0.900138 * \text{\num{1.66054e-27}}\ kg = -\text{\num{1.49876e-27}}\ kg
\end{equation}

Considering a stationary nucleus, we can use $E = mc^2$ to obtain the corresponding binding energy from the mass defect.

\begin{equation}
E = \text{\num{1.49876e-27}} * (\text{\num{3e8}})^2\ kg.m^2.s^{-2}
\end{equation}


\begin{equation}
E = \text{\num{1.34889e-10}}\ J = \text{\num{1.34889e-10}} * \text{\num{6.242e12}}\ {MeV} = 841.977\ {MeV}
\end{equation}

The binding energy per nucleon is thus $\frac{841.977}{232} = 3.629\ {MeV}$.

In reality, the mass of Th-232 is closer to $232.038\ amu$, thus giving roughly a factor 2 for the binding energy per nucleon (around $7.5\ MeV$).

\section{Spontaneous fission of Cf-252}
\label{prob12}

\subsection{Problem}
\textit{Calculate the energy released (in MeV) by the spontaneous fission of Cf-252 to produce (a) 2 equal-sized fission fragments (In-125) and 2 neutrons, (b) 2 unequal-sized fission fragments (Xe-140 and Ru-110) and 2 neutrons. Which fission reaction has a lower coulombic barrier?}

\subsection{Solution}

The binding energy difference between the reactants and the products will give the energy released by a fission event.



\begin{equation}
E = \Delta mass * 931.5\ Mev/amu = (M_{reactants} - M_{products}) * 931.5
\end{equation}


\begin{equation}
E = (252.081626 - 2*124.9136 - 2*1.008644) * 931.5\ {Mev} = 220.9\ MeV
\end{equation}

The energy released per fission event in the case if the products are two In-125 nuclides is $220.9\ {MeV}$.


Similarly, one can compute the energy released per fission event if Xe-140 and Ru-110 are the fission products, along with two neutrons.

\begin{equation}
E = (252.081626 - 139.92164 - 109.91414 - 2*1.008644) * 931.5\ {Mev} = 212.9\ {MeV}
\end{equation}


Coulomb potential $E_c$ is given, for case (a), by:

\begin{equation}
E_c = \frac{e^2Z_{In}^2}{4\pi \varepsilon_0 r}
\end{equation}

If the distance $r$ between the nucleus is reduced to its minimum, we have $r = 2*r_{In} = 2*r_0*A_{In}^{1/3}$.

Here, $\varepsilon_0$ is the vacuum permittivity, $\varepsilon_0 = \text{\num{8.854e-17}}$.

For case (a), we thus have:
\begin{equation}
E_c = \frac{\text{\num{1.6021e-19}}^2 * 49^2}{4\pi \text{\num{8.854e-12}} 2*\text{\num{1.2e-15}}*125^{1/3}} * \text{\num{6.242e12}}\ {MeV}
\end{equation}

Consequently, $E_{c,a} = 288.1\ {MeV}$.

For case (b), we have:

\begin{equation}
E_c = \frac{\text{\num{1.6021e-19}}^2 * 54 * 44}{4\pi \text{\num{8.854e-12}} \text{\num{1.2e-15}}*\left( 140^{1/3} + 110^{1/3} \right)} * \text{\num{6.242e12}}\ {MeV}
\end{equation}

Consequently, $E_{c,b} = 285.6\ {MeV}$.

Case (b) has a lower Coulomb barrier to overcome in order to give a spontaneous fission. However, its stationary fission energy release is also lower.


\section{Energy release}
\label{prob13}


\subsection{Problem}
\textit{Compare the energy released (MeV per gram of reactant) by the fusion of deuterium to give helium-4 to the energy released in the fission of U-235 by a slow neutron to give Kr-90, Ba-144, and 2 neutrons. The nuclear reactions are ${}^2\textrm{D} + {}^2\textrm{D} \to {}^{4}\textrm{He}$ and ${}^{235}\textrm{U} + {}^1\textrm{n} \to {}^{90}\textrm{Kr} + {}^{144}\textrm{Ba} + 2{}^1\textrm{n}$.}

\subsection{Solution}

\begin{equation}
{}^2\textrm{D} + {}^2\textrm{D} \to {}^{4}\textrm{He} + {}^1\textrm{n}
\end{equation}


\begin{equation}
Q = 4*7.074 - (2*1.112+3*2.827) = 17.589\ {MeV}
\end{equation}

One D-D fusion reaction liberates 17.589 MeV.

The reactant have a combined paired atomic mass number of 4. We can thus have $N_A / 4 = \text{\num{1.5e23}}$ pairs of reactants in 1 gram. Consequently, 1 g of Deuterium can produce $\text{\num{2.65e24}}\ {MeV} = \text{4.24e11}\ J = 117.8\ {MWh}$.

In the meantime, for the fission, we have:

\begin{equation}
Q = 90*8.591 + 144*8.265 - 235*7.591
\end{equation}

One U-235 fission with given fission products liberates 179.5 MeV.

We can have $N_A / 236 = \text{\num{2.55e21}}$ pairs of reactants in 1 gram. Consequently, 1 g of pure Uranium-235 can produce $\text{\num{4.6e23}}\ {MeV} = \text{7.33e10}\ J = 20.4\ {MWh}$.
